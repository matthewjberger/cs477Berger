\documentclass[14pt]{article}

\usepackage{fancyhdr}
\usepackage{extramarks}
\usepackage{amsmath}
\usepackage{amsthm}
\usepackage{amsfonts}
\usepackage{amssymb}
\usepackage{tikz}
\usepackage[plain]{algorithm}
\usepackage{algpseudocode}
\usepackage{enumitem}
\usepackage{relsize}
\usepackage{scrextend}
\usepackage{listings}
\usepackage{xcolor}
\usepackage{textcomp}

\usetikzlibrary{automata,positioning}

%
% C++ Code Listing Configuration
%
\definecolor{listinggray}{gray}{0.9}
\definecolor{lbcolor}{rgb}{0.9,0.9,0.9}
\lstset{
backgroundcolor=\color{lbcolor},
    tabsize=4,
%   rulecolor=,
    language=[GNU]C++,
        basicstyle=\scriptsize,
        upquote=true,
        aboveskip={1.5\baselineskip},
        columns=fixed,
        showstringspaces=false,
        extendedchars=false,
        breaklines=true,
        prebreak = \raisebox{0ex}[0ex][0ex]{\ensuremath{\hookleftarrow}},
        frame=single,
        numbers=left,
        showtabs=false,
        showspaces=false,
        showstringspaces=false,
        identifierstyle=\ttfamily,
        keywordstyle=\color[rgb]{0,0,1},
        commentstyle=\color[rgb]{0.026,0.112,0.095},
        stringstyle=\color[rgb]{0.627,0.126,0.941},
        numberstyle=\color[rgb]{0.205, 0.142, 0.73},
%        \lstdefinestyle{C++}{language=C++,style=numbers}’.
}
\lstset{
    backgroundcolor=\color{lbcolor},
    tabsize=4,
  language=C++,
  captionpos=b,
  tabsize=3,
  frame=lines,
  numbers=left,
  numberstyle=\tiny,
  numbersep=5pt,
  breaklines=true,
  showstringspaces=false,
  basicstyle=\footnotesize,
%  identifierstyle=\color{magenta},
  keywordstyle=\color[rgb]{0,0,1},
  commentstyle=\color{Darkgreen},
  stringstyle=\color{red}
  }

%
% Basic Document Settings
%

\topmargin=-0.45in
\evensidemargin=0in
\oddsidemargin=0in
\textwidth=6.5in
\textheight=9.0in
\headsep=0.25in

\linespread{1.1}

\pagestyle{fancy}
\lhead{\hmwkAuthorName}
\chead{\hmwkClass\ (\hmwkClassInstructor): \hmwkTitle}
\rhead{\firstxmark}
\lfoot{\lastxmark}
\cfoot{\thepage}

\renewcommand\headrulewidth{0.4pt}
\renewcommand\footrulewidth{0.4pt}

\setlength\parindent{0pt}

%
% Create Problem Sections
%

\newcommand{\enterProblemHeader}[1]{
    \nobreak\extramarks{}{Problem \arabic{#1} continued on next page\ldots}\nobreak{}
    \nobreak\extramarks{Problem \arabic{#1} (continued)}{Problem \arabic{#1} continued on next page\ldots}\nobreak{}
}

\newcommand{\exitProblemHeader}[1]{
    \nobreak\extramarks{Problem \arabic{#1} (continued)}{Problem \arabic{#1} continued on next page\ldots}\nobreak{}
    \stepcounter{#1}
    \nobreak\extramarks{Problem \arabic{#1}}{}\nobreak{}
}

\setcounter{secnumdepth}{0}
\newcounter{partCounter}
\newcounter{homeworkProblemCounter}
\setcounter{homeworkProblemCounter}{1}
\nobreak\extramarks{Problem \arabic{homeworkProblemCounter}}{}\nobreak{}

%
% Homework Problem Environment
%
% This environment takes an optional argument. When given, it will adjust the
% problem counter. This is useful for when the problems given for your
% assignment aren't sequential. See the last 3 problems of this template for an
% example.
%
\newenvironment{homeworkProblem}[1][-1]{
    \ifnum#1>0
        \setcounter{homeworkProblemCounter}{#1}
    \fi
    \section{Problem \arabic{homeworkProblemCounter}}
    \setcounter{partCounter}{1}
    \enterProblemHeader{homeworkProblemCounter}
}{
    \exitProblemHeader{homeworkProblemCounter}
}

%
% Homework Details
%   - Title
%   - Due date
%   - Class
%   - Section/Time
%   - Instructor
%   - Author
%

\newcommand{\hmwkTitle}{Homework\ \#5}
\newcommand{\hmwkDueDate}{November 3rd, 2016}
\newcommand{\hmwkDueTime}{2:30pm}
\newcommand{\hmwkClass}{CS 477}
\newcommand{\hmwkClassInstructor}{Monica Nicolescu}
\newcommand{\hmwkAuthorName}{Matthew J. Berger}

%
% Title Page
%

\title{
    \vspace{2in}
    \textmd{\textbf{\hmwkClass:\ \hmwkTitle}}\\
    \normalsize\vspace{0.1in}\small{Due\ on\ \hmwkDueDate\ at \hmwkDueTime}\\
    \vspace{0.1in}\large{\textit{\hmwkClassInstructor}}
    \vspace{3in}
}

\author{\textbf{\hmwkAuthorName}}
\date{}

\renewcommand{\part}[1]{\textbf{\large Part \Alph{partCounter}}\stepcounter{partCounter}\\}

%
% Various Helper Commands
%

% Useful for algorithms
\newcommand{\alg}[1]{\textsc{\bfseries \footnotesize #1}}

% For derivatives
\newcommand{\deriv}[1]{\frac{\mathrm{d}}{\mathrm{d}x} (#1)}

% For partial derivatives
\newcommand{\pderiv}[2]{\frac{\partial}{\partial #1} (#2)}

% Integral dx
\newcommand{\dx}{\mathrm{d}x}

% Alias for the Solution section header
\newcommand{\solution}{\textbf{\large Solution}}

% Alias for the Extra Credit section header
\newcommand{\extracredit}{\textbf{\large Extra Credit}}

% Alias for the Big-O Symbol
\newcommand{\bigo}{\mathcal{O}}

% Probability commands: Expectation, Variance, Covariance, Bias
\newcommand{\E}{\mathrm{E}}
\newcommand{\Var}{\mathrm{Var}}
\newcommand{\Cov}{\mathrm{Cov}}
\newcommand{\Bias}{\mathrm{Bias}}

\begin{document}

\maketitle

\pagebreak

\begin{homeworkProblem}

	\begin{itemize}
		\item[\textbf{1.)}] \textbf{(U \& G Required)[100 points]}

		Suppose you are consulting for a company that manufactures PC equipment and ships it
to distributors all over the country. For each of the n next weeks, they have a projected
supply si of equipment (measured in pounds), which has to be shipped by an air freight
carrier. Each week’s supply can be carried by one of two air freight companies, A or B.
		\begin{itemize}
			\item Company A charges a fixed rate r per pound, so it costs $r * s_i$ to ship a week’s
supply ($s_i$)
			\item Company B makes contracts for a fixed amount c per week, independent of the
weight. However, contracts with company B must be made in blocks of four
consecutive weeks at a time.
		\end{itemize}

		A schedule, for the PC company, is a choice of air freight company (A or B) for each of
the n weeks with the restriction that company B, whenever it is chosen, must be chosen
for blocks of four contiguous weeks at a time. The cost of the schedule is the total amount
paid to companies A and B, according to the description above.

You are asked to give a polynomial time algorithm that takes a sequence of supply values
$s_1, s_2, \dots, s_n$ and returns a schedule of minimum cost. In order to achieve this, you need to
answer the following questions:

		\begin{enumerate}[label=\alph*.)]
			\item\ [20 points] Determine and \textbf{prove} the optimal substructure of the problem and write a recursive formula of an optimal solution (i.e., define the variable that you wish to
optimize and explain how a solution to computing it can be obtained from solutions to
subproblems). \\
\textbf{Submit}: the recursive formula, along with definitions and explanations on
what is computed.\\

\solution\\
We'll call the algorithm $MINCOST(i)$. In this case, the algorithm computes the lowest cost possible to ship the the PC equipment to distributors for the first $i$ weeks. We'll also declare an algorithm called $OPTIMIZE(i,j)$ to find the company for the j-th week that would achieve $MINCOST(i)$. The best schedule for the first $i$ weeks will either be acquired through choosing company A for the i-th week (a single week) or by choosing company B for the previous 3 weeks as well as the i-th one (weeks $i,i-1,i-2,$ and $i-3$).

The optimal substructure of this problem can be represented by the equation below:

			\item[(1)]\hfil \( MINCOST(i) = min\{MINCOST(i-1) + r *s_i,\ MINCOST(i-4)+4c\} \) for $i \geq 4$

However for $i < 4$, we must choose company A for each week:
			\item[(2)]\hfil \( MINCOST(0) = 0\)
			\item[(3)]\hfil \( MINCOST(i) = MINCOST(i-1)+r*s_i\)
			\item\ [30 points] Write an algorithm that computes an optimal solution to this problem, based on the recurrence above. Implement your algorithm in C/C++ and run it on the
following values:
			\begin{itemize}
				\item $r = 1$
				\item $c = 10$
				\item the sequence of $s_i$ values: 11, 9, 9, 12, 12, 12, 12, 9, 9, 11
			\end{itemize}
			\textbf{Submit}:
			\begin{itemize}
				\item A printed version of the algorithm (name your algorithm schedule.c or schedule.cpp)
				\item A printout of the table that contains the solutions to the subproblems, run on the
values given above (print the entire table!)
			\end{itemize}
			\solution
\begin{lstlisting}
#include <algorithm>

#define NUM_WEEKS 10
#define COMPANY_A 0
#define COMPANY_B 1

int main()
{
    int r = 1;
    int s[NUM_WEEKS] = { 11, 9, 9, 12, 12, 12, 12, 9, 9, 11 };
    int c = 1;
    int minCost[NUM_WEEKS] = {0};

    int opt[NUM_WEEKS][NUM_WEEKS] = {{0}};

    for(int i = 1; i < NUM_WEEKS; i++)
    {
        if(i < 4)
        {
            minCost[i] = minCost[i-1] + (r * s[i]);

            for(int j = 1; j < i-1; j++)
            {
                opt[i][j] = opt[i-1][j];
            }
            opt[i][i] = COMPANY_A;
        }
        else
        {
            
            int costA = minCost[i-1] + (r * s[i]);
            int costB = minCost[i-4] + (4 * c);
            minCost[i] = std::min(costA,costB);

            if(costA < costB)
            {
                for(int j = 1; j < i-1; j++)
                {
                    opt[i][j] = opt[i-1][j];
                }
                opt[i][i] = COMPANY_A;
            }
            else
            {
                for(int j = 1; j < i-4; j++)
                {
                    opt[i][j] = opt[i-4][j];
                }

                for(int x = 3; x >= 0; x--)
                {
                    opt[i][i-x] = COMPANY_B;
                }
            }
        }
    }

    for(int i = 0; i < NUM_WEEKS; i++)
    {
        std::string choice = (opt[i][i] == COMPANY_A) ? "Company A" : "Company B";
        std::cout << "Week " << i << ": " << choice << std::endl;
    }

    return 0;
}

\end{lstlisting}
			\item\ [20 points] Update the algorithm you developed at point (b) to enable the
reconstruction of the optimal solution, i.e., which company was used in an optimal
solution for shipping. (Hint: use an auxiliary table like we did in the examples in class.)
Include these updates in your algorithm implementation from point (b).\\
			\textbf{Submit}:
			\begin{itemize}
				\item A printed version of the algorithm (name your algorithm schedule\_1.c or
schedule\_1.cpp).
				\item A printout of the values that you obtain in the table containing the additional information needed to reconstruct the optimal solution, run on the values given
above (print the entire table!)
			\end{itemize}
			\item\ [30 points] Using the additional information computed at point (c), write an algorithm that outputs which company was used for shipping in the optimal schedule. Implement
this algorithm in C/C++. \\
			\textbf{Submit}:
			\begin{itemize}
				\item A printed version of the algorithm (name your algorithm schedule\_2.c or
schedule\_2.cpp).
				\item A printout of the \textbf{solution} to the problem, i.e., the optimal schedule. (e.g., A, A,
B, A, B)
			\end{itemize}
		\end{enumerate}
	\end{itemize}
\end{homeworkProblem}
\end{document}
