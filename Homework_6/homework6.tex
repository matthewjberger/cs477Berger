\documentclass{article}

\usepackage{fancyhdr}
\usepackage{extramarks}
\usepackage{amsmath}
\usepackage{amsthm}
\usepackage{amsfonts}
\usepackage{amssymb}
\usepackage{tikz}
\usepackage[plain]{algorithm}
\usepackage{algpseudocode}
\usepackage{enumitem}
\usepackage{relsize}
\usepackage{scrextend}
\usepackage{listings}
\usepackage{xcolor}
\usepackage{textcomp}

\usetikzlibrary{automata,positioning}

%
% C++ Code Listing Configuration
%
\definecolor{listinggray}{gray}{0.9}
\definecolor{lbcolor}{rgb}{0.9,0.9,0.9}
\lstset{
backgroundcolor=\color{lbcolor},
    tabsize=4,
%   rulecolor=,
    language=[GNU]C++,
        basicstyle=\scriptsize,
        upquote=true,
        aboveskip={1.5\baselineskip},
        columns=fixed,
        showstringspaces=false,
        extendedchars=false,
        breaklines=true,
        prebreak = \raisebox{0ex}[0ex][0ex]{\ensuremath{\hookleftarrow}},
        frame=single,
        numbers=left,
        showtabs=false,
        showspaces=false,
        showstringspaces=false,
        identifierstyle=\ttfamily,
        keywordstyle=\color[rgb]{0,0,1},
        commentstyle=\color[rgb]{0.026,0.112,0.095},
        stringstyle=\color[rgb]{0.627,0.126,0.941},
        numberstyle=\color[rgb]{0.205, 0.142, 0.73},
%        \lstdefinestyle{C++}{language=C++,style=numbers}’.
}
\lstset{
    backgroundcolor=\color{lbcolor},
    tabsize=4,
  language=C++,
  captionpos=b,
  tabsize=3,
  frame=lines,
  numbers=left,
  numberstyle=\tiny,
  numbersep=5pt,
  breaklines=true,
  showstringspaces=false,
  basicstyle=\footnotesize,
%  identifierstyle=\color{magenta},
  keywordstyle=\color[rgb]{0,0,1},
  commentstyle=\color{Darkgreen},
  stringstyle=\color{red}
  }

%
% Basic Document Settings
%

\topmargin=-0.45in
\evensidemargin=0in
\oddsidemargin=0in
\textwidth=6.5in
\textheight=9.0in
\headsep=0.25in

\linespread{1.1}

\pagestyle{fancy}
\lhead{\hmwkAuthorName}
\chead{\hmwkClass\ (\hmwkClassInstructor): \hmwkTitle}
\rhead{\firstxmark}
\lfoot{\lastxmark}
\cfoot{\thepage}

\renewcommand\headrulewidth{0.4pt}
\renewcommand\footrulewidth{0.4pt}

\setlength\parindent{0pt}

%
% Create Problem Sections
%

\newcommand{\enterProblemHeader}[1]{
    \nobreak\extramarks{}{Problem \arabic{#1} continued on next page\ldots}\nobreak{}
    \nobreak\extramarks{Problem \arabic{#1} (continued)}{Problem \arabic{#1} continued on next page\ldots}\nobreak{}
}

\newcommand{\exitProblemHeader}[1]{
    \nobreak\extramarks{Problem \arabic{#1} (continued)}{Problem \arabic{#1} continued on next page\ldots}\nobreak{}
    \stepcounter{#1}
    \nobreak\extramarks{Problem \arabic{#1}}{}\nobreak{}
}

\setcounter{secnumdepth}{0}
\newcounter{partCounter}
\newcounter{homeworkProblemCounter}
\setcounter{homeworkProblemCounter}{1}
\nobreak\extramarks{Problem \arabic{homeworkProblemCounter}}{}\nobreak{}

%
% Homework Problem Environment
%
% This environment takes an optional argument. When given, it will adjust the
% problem counter. This is useful for when the problems given for your
% assignment aren't sequential. See the last 3 problems of this template for an
% example.
%
\newenvironment{homeworkProblem}[1][-1]{
    \ifnum#1>0
        \setcounter{homeworkProblemCounter}{#1}
    \fi
    \section{Problem \arabic{homeworkProblemCounter}}
    \setcounter{partCounter}{1}
    \enterProblemHeader{homeworkProblemCounter}
}{
    \exitProblemHeader{homeworkProblemCounter}
}

%
% Homework Details
%   - Title
%   - Due date
%   - Class
%   - Section/Time
%   - Instructor
%   - Author
%

\newcommand{\hmwkTitle}{Homework\ \#6}
\newcommand{\hmwkDueDate}{November 17th, 2016}
\newcommand{\hmwkDueTime}{2:30pm}
\newcommand{\hmwkClass}{CS 477}
\newcommand{\hmwkClassInstructor}{Monica Nicolescu}
\newcommand{\hmwkAuthorName}{Matthew J. Berger}

%
% Title Page
%

\title{
    \vspace{2in}
    \textmd{\textbf{\hmwkClass:\ \hmwkTitle}}\\
    \normalsize\vspace{0.1in}\small{Due\ on\ \hmwkDueDate\ at \hmwkDueTime}\\
    \vspace{0.1in}\large{\textit{\hmwkClassInstructor}}
    \vspace{3in}
}

\author{\textbf{\hmwkAuthorName}}
\date{}

\renewcommand{\part}[1]{\textbf{\large Part \Alph{partCounter}}\stepcounter{partCounter}\\}

%
% Various Helper Commands
%

% Useful for algorithms
\newcommand{\alg}[1]{\textsc{\bfseries \footnotesize #1}}

% For derivatives
\newcommand{\deriv}[1]{\frac{\mathrm{d}}{\mathrm{d}x} (#1)}

% For partial derivatives
\newcommand{\pderiv}[2]{\frac{\partial}{\partial #1} (#2)}

% Integral dx
\newcommand{\dx}{\mathrm{d}x}

% Alias for the Solution section header
\newcommand{\solution}{\textbf{\large Solution}}

% Alias for the Extra Credit section header
\newcommand{\extracredit}{\textbf{\large Extra Credit}}

% Alias for the Big-O Symbol
\newcommand{\bigo}{\mathcal{O}}

% Probability commands: Expectation, Variance, Covariance, Bias
\newcommand{\E}{\mathrm{E}}
\newcommand{\Var}{\mathrm{Var}}
\newcommand{\Cov}{\mathrm{Cov}}
\newcommand{\Bias}{\mathrm{Bias}}

\begin{document}

\maketitle

\pagebreak

\begin{homeworkProblem}
	\begin{itemize}
		\item[\textbf{1.)}] \textbf{[40 points](U \& G Required)}\\
	A company with a photocopying service that has a single large machine faces the following scheduling problem. Every morning, the company receives a set of jobs from its customers. Ideally, the jobs should be run on their single machine in an order that keeps their customers happiest. We know that customer $i$'s job will take $t_i$ time to complete. Given a schedule (i.e., an ordering of the jobs), let $C_i$ represent the finishing time of job $i$. For example, if job $i$ is the first to be done, we would have $C_i$ = $t_i$; and if job $j$ is done right after job $i$, we would have $C_j$ = $C_i$ + $t_j$. Each customer $i$ also has a given weight $w_i$ that represents his or her importance to the business. The happiness of customer $i$ is expected to be dependent on the finishing time of $i$'s job. Therefore, the company decides that they want to order the jobs to minimize the weighted sum of the completion times \(\mathlarger{\sum\limits_{i=1}^n w_iC_i}\).\\
	Design an efficient algorithm to solve this problem: given a set of $n$ jobs with a processing time $t_i$ and a weight $w_i$ for each job, order the jobs so as to minimize the weighted sum of the completion times, \(\mathlarger{\sum\limits_{i=1}^n w_iC_i}\).\\\\
	\textbf{Example:} Suppose there are two jobs: the first takes time $t_1 = 1$ and has weight $w_1 = 10$, while the second job takes time $t_2 = 3$ and has weight $w_2 = 2$. Then doing job 1 first would yield a weighted completion time of $10*1 + 2*4 = 18$, while doing the second job first would yield the larger weighted completion time of $10*4 + 2*3 = 46$.
Note: you have to prove that your strategy yields the optimal solution.

	\item[] 
	\solution
\begin{algorithm}
  \caption{MinimizeWSCT}\label{MWSCT}
  \begin{algorithmic}[1]
    \Procedure{MinimizeWSCT}{$a,b$}\Comment{Minimize Weighted Sum of Completion Times}
      \State $weightsList \gets <the\ weight\ of\ each\ job>$
      \State $timesList \gets <the\ processing\ time\ for\ each\ job>$
      \State $timeWeightRatioMap \gets <a map time to weight for\ each job>$
      \State $tmpTime\gets 0$
      \State $sum\gets 0$
      \While{$ratios.Size() > 0$}\Comment{Iterate through all ratios in map}
        \State $tmpTime += timesList.get(timeWeightRatioMap.Keys[0])$
        \State $sum += weightsList.get(timeWeightRatioMap.Keys[0]) * tmpTime$
        \State $timeWeightRatioMap.Remove(timeWeightRatioMap.Keys[0])$
      \EndWhile\label{MWSCTendwhile}
    \EndProcedure
  \end{algorithmic}
\end{algorithm}
\hfil
	\end{itemize}
\end{homeworkProblem}
\pagebreak
\begin{homeworkProblem}
	\begin{itemize}
		\item[\textbf{2.)}] \textbf{[40 points](U \& G Required)}\\
		A student camp counselor is in charge of organizing activities for a set of junior-highschool-age
campers. He plans the following mini-triathlon exercise: each contestant must
swim 20 laps of a pool, then bike 10 miles, then run 3 miles. His plan is to send the 
contestants out in a staggered fashion, based on the following rule: the contestants must
use the pool one at a time. In other words, the first contestant swims the 20 laps, gets out,
and starts biking. As soon as the first person is out of the pool, a second contestant begins
swimming the 20 laps; as soon as he or she is out and starts biking, a third contestant
begins swimming … and so on.).
Each contestant has a projected \textit{swimming time} (the expected time it will take him or her
to complete the 20 laps), a projected \textit{biking time} (the expected time it will take him or her
to complete the 10 miles of bicycling), and a projected \textit{running time} (the time it will take
him or her to complete the 3 miles of running). The camp counselor wants to decide on a
\textit{schedule} for the triathlon: an order in which to sequence the starts of the contestants.
Let’s say that the completion time of a schedule is the earliest time at which all
contestants will be finished with all three legs of the triathlon, assuming they each spend
exactly their projected swimming, biking, and running times on the three parts. (Again,
note that participants can bike and run simultaneously, but at most one person can be in
the pool at any time.) What’s the best order for sending people out, if one wants the
whole competition to be over as early as possible? More precisely, give an efficient
algorithm that produces a schedule whose completion time is as small as possible. Note: you have to prove that your strategy yields the optimal solution.
	\end{itemize}
\end{homeworkProblem}
\pagebreak
\begin{homeworkProblem}
	\begin{itemize}
		\item[\textbf{3.)}] \textbf{[20 points](U \& G Required)}\\
		Construct an optimal Huffman coding tree for the characters I with frequency 75, U with
frequency 200, B with frequency 25, S with frequency 275, C with frequency 50, H with
frequency 100, M with frequency 25, and P with frequency 250. How many bits are
required to store the above characters using the resulting encoding? (Show your work for
building the Huffman codes).
	\end{itemize}
\end{homeworkProblem}
\pagebreak
\begin{homeworkProblem}[5]
	\begin{itemize}
		\item[\textbf{5.)}] \textbf{[20 points](Extra Credit)}\\
Consider a long, quiet country road with houses scattered very sparsely along it. (We can
picture the road as a long line segment, with an eastern endpoint and a western endpoint).
The residents of all these houses are avid cell phone users. You want to place cell phone
base stations at certain points along the road, so that every house is within four miles of
one of the base stations. Give an efficient (greedy) algorithm that achieves this goal,
using as few base stations as possible. Note: you have to prove that your strategy yields
the optimal solution.
	\end{itemize}
	
	\solution
	
	A possible greedy algorithm to solve this would be one that starts from the leftmost point and at each step places the beginning of the interval at the first uncovered point. This is equivalent to covering a set of points on a real line with 8-unit length intervals, with the goal being to use the minimum number of intervals.
	
	To prove this, we can use contradiction. Greedy is not optimal. Let $g_1 < g_2 < ... < g_n$ represent the set of starting positions indicated by the intervals chosen in the algorithm. Let $f_1 < f_2 < ... < f_n$ denote the set of starting points at the intervals of the optimal solution with $f_1 = g_1, f_2 = g_2, \dots, f_n=g_n$ for the largest possible value of n. $g_{n+1} > f_{n+1} $ because of the greedy algorithm. The optimal solution is modified further by making $f'_{n+1} = g_{n+1}$ (pushing the $(r+1)_{th}$ interval further). This covers all of the points, because there cannot be any points in $(f_{n+1},g_{n+1})$ due to the greedy algorithm. The number of intervals does not change, therefore it is optimal. The contradiction is that the new optimal solution has one more starting point in common with the greedy solution.
\end{homeworkProblem}
\end{document}