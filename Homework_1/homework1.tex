\documentclass[14pt]{article}

\usepackage{fancyhdr}
\usepackage{extramarks}
\usepackage{amsmath}
\usepackage{amsthm}
\usepackage{amsfonts}
\usepackage{tikz}
\usepackage[plain]{algorithm}
\usepackage{algpseudocode}
\usepackage{enumitem}
\usepackage{relsize}
\usepackage{scrextend}

\usetikzlibrary{automata,positioning}

%
% Basic Document Settings
%

\topmargin=-0.45in
\evensidemargin=0in
\oddsidemargin=0in
\textwidth=6.5in
\textheight=9.0in
\headsep=0.25in

\linespread{1.1}

\pagestyle{fancy}
\lhead{\hmwkAuthorName}
\chead{\hmwkClass\ (\hmwkClassInstructor): \hmwkTitle}
\rhead{\firstxmark}
\lfoot{\lastxmark}
\cfoot{\thepage}

\renewcommand\headrulewidth{0.4pt}
\renewcommand\footrulewidth{0.4pt}

\setlength\parindent{0pt}

%
% Create Problem Sections
%

\newcommand{\enterProblemHeader}[1]{
    \nobreak\extramarks{}{Problem \arabic{#1} continued on next page\ldots}\nobreak{}
    \nobreak\extramarks{Problem \arabic{#1} (continued)}{Problem \arabic{#1} continued on next page\ldots}\nobreak{}
}

\newcommand{\exitProblemHeader}[1]{
    \nobreak\extramarks{Problem \arabic{#1} (continued)}{Problem \arabic{#1} continued on next page\ldots}\nobreak{}
    \stepcounter{#1}
    \nobreak\extramarks{Problem \arabic{#1}}{}\nobreak{}
}

\setcounter{secnumdepth}{0}
\newcounter{partCounter}
\newcounter{homeworkProblemCounter}
\setcounter{homeworkProblemCounter}{1}
\nobreak\extramarks{Problem \arabic{homeworkProblemCounter}}{}\nobreak{}

%
% Homework Problem Environment
%
% This environment takes an optional argument. When given, it will adjust the
% problem counter. This is useful for when the problems given for your
% assignment aren't sequential. See the last 3 problems of this template for an
% example.
%
\newenvironment{homeworkProblem}[1][-1]{
    \ifnum#1>0
        \setcounter{homeworkProblemCounter}{#1}
    \fi
    \section{Problem \arabic{homeworkProblemCounter}}
    \setcounter{partCounter}{1}
    \enterProblemHeader{homeworkProblemCounter}
}{
    \exitProblemHeader{homeworkProblemCounter}
}

%
% Homework Details
%   - Title
%   - Due date
%   - Class
%   - Section/Time
%   - Instructor
%   - Author
%

\newcommand{\hmwkTitle}{Homework\ \#1}
\newcommand{\hmwkDueDate}{September 13, 2016}
\newcommand{\hmwkDueTime}{2:30pm}
\newcommand{\hmwkClass}{CS 477}
\newcommand{\hmwkClassInstructor}{Monica Nicolescu}
\newcommand{\hmwkAuthorName}{Matthew J. Berger}

%
% Title Page
%

\title{
    \vspace{2in}
    \textmd{\textbf{\hmwkClass:\ \hmwkTitle}}\\
    \normalsize\vspace{0.1in}\small{Due\ on\ \hmwkDueDate\ at \hmwkDueTime}\\
    \vspace{0.1in}\large{\textit{\hmwkClassInstructor}}
    \vspace{3in}
}

\author{\textbf{\hmwkAuthorName}}
\date{}

\renewcommand{\part}[1]{\textbf{\large Part \Alph{partCounter}}\stepcounter{partCounter}\\}

%
% Various Helper Commands
%

% Useful for algorithms
\newcommand{\alg}[1]{\textsc{\bfseries \footnotesize #1}}

% For derivatives
\newcommand{\deriv}[1]{\frac{\mathrm{d}}{\mathrm{d}x} (#1)}

% For partial derivatives
\newcommand{\pderiv}[2]{\frac{\partial}{\partial #1} (#2)}

% Integral dx
\newcommand{\dx}{\mathrm{d}x}

% Alias for the Solution section header
\newcommand{\solution}{\textbf{\large Solution}}

% Alias for the Big-O Symbol
\newcommand{\bigo}{\mathcal{O}}

% Probability commands: Expectation, Variance, Covariance, Bias
\newcommand{\E}{\mathrm{E}}
\newcommand{\Var}{\mathrm{Var}}
\newcommand{\Cov}{\mathrm{Cov}}
\newcommand{\Bias}{\mathrm{Bias}}

\begin{document}

\maketitle

\pagebreak

\begin{homeworkProblem}
    \textbf{(U \& G-required)[30 points]} Arrange the following list of functions in ascending order of growth rate. That is, if function g(n) immediately follows function f(n) in your list.  

    \begin{itemize}       
        \item[] \(f_1(n) = n^{4.5}\)
        \item[] \(f_2(n) = \sqrt{2n^2 + 1}\)
        \item[] \(f_3(n) = n^2 + 10\)
        \item[] \(f_4(n) = 10^n\)
        \item[] \(f_5(n) = 100^n\)
        \item[] \(f_6(n) = n^2\log{n}\)
    \end{itemize}

    \textbf{Solution:}
    
    \begin{table}[h]
    \def\arraystretch{1.5}% 
    \begin{tabular}{|l|l|}
        \hline
        \multicolumn{1}{|c|}{Formula} & \multicolumn{1}{|c|}{Complexity} \\
        \hline
        \(f_2(n) = \sqrt{2n^2 + 1} \) & \(\bigo(n)\)          \\ 
        \(f_6(n) = n^2\log{n} \)      & \(\bigo(n^2\log{n})\) \\ 
        \(f_3(n) = n^2 + 10 \)        & \(\bigo(n^2)\)        \\ 
        \(f_1(n) = n^{4.5} \)         & \(\bigo(n^{4.5})\)    \\ 
        \(f_4(n) = 10^n \)            & \(\bigo(10^n)\)       \\ 
        \(f_5(n) = 100^n \)           & \(\bigo(100^n)\)      \\                                            
        \hline
    \end{tabular}
\end{table}
    
\end{homeworkProblem}

\pagebreak

\begin{homeworkProblem}
    \textbf{(U \& G-required)[30 points]} Using mathematical induction, show that the following relations are true for every $n \geq 1$:
    
    \begin{enumerate}[label=\alph*)]
    	\item \(\mathlarger{\sum\limits_{i=1}^n i^3=\left[\frac{n(n+1)}{2}\right]^2}\)
    	\item \(\mathlarger{\sum\limits_{i=1}^ni(i+1)=\frac{n(n+1)(n+2)}{3}}\)
    \end{enumerate}
    
    \textbf{Solution:}
    
    \textbf{Part A}
    \begin{addmargin}[2em]{2em}
    \textsl{\textbf{\underline{Proof:}} We will prove by induction that, for all n $\in \mathbb{Z_+}$,}
    
    \begin{itemize}
    	\item[(1)] \(\hfil\mathlarger{\sum\limits_{i=1}^n i^3=\left[\frac{n(n+1)}{2}\right]^2}\)
    \end{itemize}
    
        \textsl{\textbf{\underline{Base Case:}} When $n = 1$, the left side of (1) is $(1)^3 = 1$, and the right side is $\left[\frac{1(1+1)}{2}\right]^2 = \left[\frac{2}{2}\right]^2 = 1$, so both sides are equal and (1) is true for $n = 1$.}
        
        \textsl{\textbf{\underline{Induction Step:}} Let $k \in \mathbb{Z_+}$ be given and suppose (1) is true for $n = k$. Then}
    \[
    	\begin{split}
    		\sum\limits_{i=1}^{k+1} i^3&=\sum\limits_{i=1}^k +\ (k+1)^3 \\
    		\left[\frac{(k+1)(k+2)}{2}\right]^2 &=\ \left[\frac{k(k+1)}{2}\right]^2 + (k+1)^3\\
    		&=\ \frac{k^2(k+1)}{2^2}^2 + (k+1)^3\\
    		&=\ \frac{k^2(k+1)}{4}^2 + \frac{4(k+1)^3}{4}\\
    		&=\ \frac{(k+1)^2(k^2 + 4k+4)}{4}\\
    		&=\ \frac{(k+1)^2(k+2)^2}{2^2}\\
    		&=\ \left[\frac{(k+1)(k+2)}{2}\right]^2\\
    	\end{split}
    \]
    \textsl{Thus, (1) holds for $n = k + 1$, and the proof of the induction step is complete.}
    
    \textsl{\textbf{\underline{Conclusion:}} By the principle of induction, (1) is true for all $n \in \mathbb{Z_+}$}
    \end{addmargin}
    
    \pagebreak
    
    \textbf{Part B}
    \begin{addmargin}[2em]{2em}
    \textsl{\textbf{\underline{Proof:}} We will prove by induction that, for all n $\in \mathbb{Z_+}$,}
    
    \begin{itemize}
    	\item[(1)] \(\hfil\mathlarger{\sum\limits_{i=1}^ni(i+1)=\frac{n(n+1)(n+2)}{3}}\)	
    \end{itemize}
    
        \textsl{\textbf{\underline{Base Case:}} When $n = 1$, the left side of (1) is $(1)(1+1) = 2$, and the right side is $\frac{(1)(1 + 1)(1 + 2)}{3} = 2$, so both sides are equal and (1) is true for $n = 1$.}
        
        \textsl{\textbf{\underline{Induction Step:}} Let $k \in \mathbb{Z_+}$ be given and suppose (1) is true for $n = k$. Then}
    \[
    	\begin{split}
    		\sum\limits_{i=1}^{k+1} i(i+1)      &= \sum\limits_{i=1}^k +\ (k+1)(k+2) \\
    		\frac{(k+1)(k+2)(k+3)}{3} &= \frac{k(k+1)(k+2)}{3} + (k+1)(k+2)\\
    		&= \frac{k(k+1)(k+2)}{3} + \frac{3(k+1)(k+2)}{3}\\
    		&= \frac{(k+1)(k+2)(k+3)}{3}\\
    	\end{split}
    \]
    \textsl{Thus, (1) holds for $n = k + 1$, and the proof of the induction step is complete.}
    
    \textsl{\textbf{\underline{Conclusion:}} By the principle of induction, (1) is true for all $n \in \mathbb{Z_+}$}
    \end{addmargin}
\end{homeworkProblem}

\pagebreak

\begin{homeworkProblem}
	\textbf{(U \& G-required)[40 points]} Indicate whether the first function of each of the following pairs has a smaller, same, or larger order of growth (to within a constant multiple) than the second function:
    
    \begin{enumerate}[label=\alph*)]
    	\item \(\mathit{(n-1)!}\ \text{and}\ \mathit{n!}\)
    	\item \(\mathit{\log_2n}\ \text{and}\ \mathit{\ln{n}}\)	
    	\item \(\mathit{2^\mathit{n-1}}\ \text{and}\ \mathit{2^n}\)
    	\item \(\mathit{\log_22n}\ \text{and}\ \mathit{\log_2n^2}\)
    \end{enumerate}
    
    \textbf{Solution}
\end{homeworkProblem}

\begin{homeworkProblem}[5]
	\textbf{[Extra credit - 20 points]} Using the formal definition of the asymptotic notations, prove the following statements:
    
    \begin{enumerate}[label=\alph*)]
    	\item \(5n^2 + 20 \in \bigo{O}(n^2)\)
    	\item \(n + 23 \in \bigo{O}(n^3)\)
    \end{enumerate}
    
    \textbf{Solution}
\end{homeworkProblem}

\end{document}
